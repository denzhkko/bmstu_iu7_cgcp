\chapter{Аналитический раздел}
\label{cha:analysis}

  \section{Формализация объектов синтезируемой сцены}
  
    Сцена состоит из следующих объектов
    
    \begin{itemize}
      \item Источник света - объект, испускающий электромагнитное излучение в видимом спектре. Рассмотрим 2 типа источника света
      \begin{itemize}
        \item Точечный источник света - материальная точка, испускающая лучи света во всех стороны
        \item Окружающее освещение - безусловные источник освещение. Тривиализация рассеянного света, отражающегося от других объектов
      \end{itemize}
      \item Поверхность - ограничивающая плоскость, под которое не располагаются видимые объекты
      \item Прозрачные объекты - трехмерная модель, пропускающая свет сквозь себя и преломляющая его. Например, призма, алмаз.
      \item Непрозрачные объекты - трехмерная модель, не пропускающая свет сквозь себя, а отражающая его.
    \end{itemize}


  \section{Анализ алгоритмов удаление невидимых линий и поверхностей}

    \subsection{Алгоритм, использующий z-буфер}
    
      Данный алгоритм работает в пространстве изображений.   
  
      \subsubsection{Идея}

        Суть алгоритма заключается в построении матрицы глубины каждого видимого пикселя изображения. Заводятся 2 двумерных массива: буфер кадра для запоминания цвета каждого пиксела и буфер глубины для запоминания глубины каждого пиксела (z-depth буфер). При работе алгоритма глубина каждого нового рисуемого пиксела сравнивается со значением в буферы глубины. Если новое значение глубины меньше значения в ячейки матрицы, значит новый пиксель находится ближе к картинной плоскости и перекрывает пиксель, находящийся в z буфере. В таком случае необходимо обновить значение цвета пиксела в буфере кадра и значение глубины пиксела в буфере глубины.
  
      \subsubsection{Преимущества}

      \begin{itemize}
        \item алгоритм прост в реализации
        \item алгоритм имеет сложность $O(n)$, где $n$ - количество рисуемых объектов сцены
      \end{itemize}
  
      \subsubsection{Недостатки}
  
      \begin{itemize}
        \item большой объем используемой памяти
        \item отсутствие возможности работы с прозрачными объектами
        \item отсутствие возможности учитывать преломление лучей света объектов сцены
      \end{itemize}
  
    \subsection{Алгоритм Робертса}

      Данный алгоритм работает в объектном пространстве.

      \subsubsection{Идея}
      
        Алгоритм Робертса работает в 2 этапа. На первом этапе удаляются невидимые ребра рассматриваемого объекта. На втором этапе каждое из видимых ребер сравнивается с каждым из оставшихся объектом сцены для определения, какая часть или части, если таковые имеются, экранируются этими объектами.

      \subsubsection{Преимущества}
      
      \begin{itemize}
        \item алгоритм использует математически простые и точные методы
      \end{itemize}
      
      \subsubsection{Недостатки}
      
      \begin{itemize}
        \item алгоритм имеет сложность $O(n^2)$, где $n$ - число объектов
        \item отсутствие возможности работы с прозрачными объектами
        \item отсутствие возможности учитывать преломление лучей света
      \end{itemize}
  
    \subsection{Алгоритм Варнока}
    
      Данный алгоритм работает в пространстве изображений.
    
      \subsubsection{Идея}
      
        Суть алгоритма Варнока заключается в разбиении картинной плоскости на части, для каждой из которых задача построения изображения может быть решена достаточно просто. Рассматривается окно и решается вопрос: является ли окно достаточно простым для построения. Если окно не является достаточно простым, то оно разбивается на части, для каждой из которых решается исходная задача.
      
      \subsubsection{Преимущества}
      
      \begin{itemize}
        \item алгоритм эффективен для простых сцен в силу малого количества делений картинной плоскости
      \end{itemize}
      
      \subsubsection{Недостатки}
      
      \begin{itemize}
        \item алгоритм неэффективен при большом количестве пересечений объектов в силу большого количества делений картинной плоскости
        \item отсутствие возможности работы с прозрачными объектами
        \item отсутствие возможности учитывать преломление лучей света
      \end{itemize}
 
    \subsection{Алгоритм обратной трассировки лучей}
    
      Данный алгоритм работает в пространстве изображений
    
      \subsubsection{Идея}
      
        Данный алгоритм имеет физическую природу. При прямой трассировке лучей испускаются лучи от источников света и отображаются только те лучи, которые попадают в камеру, большая часть вычислений не используется. При обратной же трассировке лучи испускаются из картинной плоскости до первого пересечений с ближайшей поверхностью. Используется закон обратимости светового луча. Согласно ему, луч света, распространившийся по определённой траектории в одном направлении, повторит свой ход в точности при распространении и в обратном направлении.

        Данный алгоритм в силу физической природы позволяет также учитывать отражения и преломления лучей

      \subsubsection{Преимущества}
      
        \begin{itemize}
          \item алгоритм является чрезвычайно параллелизуемым
          \item возможность работы с прозрачными объектами
          \item возможность учитывания преломления света
        \end{itemize}

      \subsubsection{Недостатки}
      
    \subsection{Вывод}
    
      Алгоритмы z-буфера, Робертса, Варконока в их классическом представлении не позволяют учитывать преломления лучей, их отражения, также они слабо приспособлены для работы с прозрачными объектами.

      Алгоритм обратной трассировки, благодаря физической природе, прекрасно приспособлен для работы с прозрачными объектами. Также возможно изменять направления луча при прохождении через прозрачную границу, что позволяет учитывать преломления лучей.

      Главным недостатком алгоритма обратной трассировки является его вычислительная сложность, так как для каждого луча необходимо выполнять все вычисления снова. Но этот его недостаток компенсируется возможностью его распараллеливания, как для частей изображения, так и для каждого пикселя отдельно.

      В силу описанных выше преимуществ и недостатков основой для алгоритма решения поставленной задачи стоит выбрать алгоритм обратной трассировки лучей.

  \section{Анализ методов закрашивания}
  
    Тело, грани которого имеют одинаковый цвет и интенсивность, воспринимается человеков как плоская фигура. Поэтому для создаются ощущения видимости объемного изображения необходимо использовать алгоритмы закрашивания
    
    \subsection{Простая закраска}
      
      \subsubsection{Идея}
      
        Вся грань закрашивается одним уровнем интенсивности, который вычисляется по закону Ламберта:
        
        \begin{equation}
          I = I_0 \times k \times cos(\alpha)
        \end{equation}
        
        где $I$ – интенсивность света в точке,
        
        $I_0$ – интенсивность источника света,
        
        $k$ – коэффициент диффузного отражения,
        
        $\alpha$ – угол падения луча.

      \subsubsection{Преимущества}
      
        \begin{itemize}
          \item простота реализации
          \item низкие требования к вычислительным мощностями
          \item хорошо подходит для граней многоугольника
        \end{itemize}
      
      \subsubsection{Недостатки}
      
        \begin{itemize}
          \item плохо подходит для фигур вращения, так как видны ребра
        \end{itemize}
        
    \subsection{Закраска по Гуро}
      
      \subsubsection{Идея}
      
        Суть метода заключается в последовательном вычислении интенсивности у каждой из граней трехмерной модели и дальнейшем определении интенсивности путем усреднения.
      
      \subsubsection{Преимущества}
      
        \begin{itemize}
          \item хорошо подходит для фигур вращения, аппроксимированных полигонами
        \end{itemize}
      
      \subsubsection{Недостатки}
      
        \begin{itemize}
          \item ребра многогранником могут стать незаметными
        \end{itemize}
        
    \subsection{Закраска по Фонгу}
      
      \subsubsection{Идея}
      
        Алгоритм похож на алгоритм закраски по Гуро, однако аппроксимируется не интенсивность света, а вектор нормали.
      
      \subsubsection{Преимущества}
      
        \begin{itemize}
          \item наиболее реалистическое изображение
        \end{itemize}
      
      \subsubsection{Недостатки}
      
        \begin{itemize}
          \item наиболее ресурсозатратный алгоритм
        \end{itemize}
    
    \subsection{Вывод}
    
      Алгоритм простой закраски дает неудовлетворительный результат. Алгоритмы закраски по Гуро и Фонду дают похожие результаты, однако алгоритм закраски по Гуро является менее ресурсозатратным. Поэтому выбор пал на алгоритм закраски по Гуро.

  \section{Анализ цветовых моделей}
 
    \subsection{Цветовая модель RGB}
    
      Закон Грассмана гласит, что восприятие хроматической составляющей цвета описывается примерно линейным законом. Впервые подобная модель была предложена Джеймсом Максвеллом в 1861 году. В модели RGB все цвета получаются путем смешивания базовых цветов (красного, зеленого, синего) в разных пропорциях. То есть цвет задается как координата в трехмерном пространстве. Поэтому часто данную модель называют цветовым кубом. Цветные изображения строятся из трех каналов, после чего они совмещаются и получается цветное изображение. Так же подобную модель имеют и глаза человека, так как колбочки, рецепторы для считывания цветов, у человека только трех типов.
    
    \subsection{Цветовая модель CMY}
    
      Данная цветовая модель используется для печати изображений. Состоит из трех цветов: cyan - голубой, magenta - пурпурный, yellow - желтый. Так, желтая бумага не отражает синий цвет. Можно сказать, что желтая бумага вычитает из белого цвета синий, именно поэтому данную модель и называют субтрактивной.

    
    \subsection{HSI}
    
      Цветовая модель HSI используется в профессиональных графических решения для художников. Человек обычно для описания цвета использует критерии отличные от цветных составляющих: цветовой фон, насыщенность, светлота. Именно этими понятиями и оперируют в данной цветовой модели для задания цветов.
    
    \subsection{Вывод}
    
      Цветовая модель CYM используются для печати, а HSI - избыточна для решения поставленной задачи. Поэтому следует использовать цветовую модель RGB.
