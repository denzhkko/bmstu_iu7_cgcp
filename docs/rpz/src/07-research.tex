\chapter{Экспериментальный раздел}

    В данном разделе будет поставлен эксперимент, в котором будут сравнены временные характеристики работы реализованного программного обеспечения в различных конфигурациях.
    
    \section{Цель эксперимента}
    
        Рассматривая схему алгоритма обратное трассировки лучей можно заметить, вычисление значения цвета каждого пикселя происходит независимо от других значений. Таким образом данный алгоритм можно распараллелить.

        Целью эксперимента является оценка временной эффективности параллельной реализации алгоритма обратной трассировки лучей.
    
    \section{Технические характеристики}

        Технические характеристики устройства, на котором выполнялось исследование:

        \begin{itemize}
        	\item процессор: Intel Core™ i5-8250U \cite{i5} CPU @ 1.60GHz;
        	\item память: 8 GiB;
        	\item операционная система: Fedora \cite{fedora} Linux \cite{linux} 21.1.4 64-bit.
        \end{itemize}
        
        Исследование проводилось на ноутбуке, включенном в сеть электропитания. Во время тестирования ноутбук был нагружен только встроенными приложениями окружения рабочего стола, окружением рабочего стола, а также непосредственно системой тестирования.
        
    \section{Описание эксперимента}

        Была реализована функция параллельного синтеза сцены. Для этого была использована библиотека \texttt{OpenMP} \cite{omp}, директива препроцессора \texttt{\#pragma omp parallel for}. Данная директива препроцессора преобразует код для выполнения итераций цикла параллельно. 

        В рамках данного эксперимента будет производиться оценка влияния размерности изображения и количества объектов сцена на время работы алгоритма. Для этого будем синтезировать квадратные изображения с размерностями равными [100, 200, 500, 1000, 2000] элементов. Количество объектов сцены будет задано равным [2, 4, 8, 16] штук.
 
    \section{Результат эксперимента}

        В таблице \ref{tbl:time} приведены экспериментально полученные значения временных характеристик работы алгоритма в зависимости от размерности синтезируемого изображения и количества объектов сцены.
        
\begin{table}[ht]
	\small
	\begin{center}
		\caption{Замеры времени для изображений с различными размерностями}
		\label{tbl:time}
		\begin{tabular}{|c|c|c|c|}
        \hline
        & & \multicolumn{2}{c|}{Время (мс)} \\
        \cline{3-4}
        \raisebox{1.5ex}{Кол-во объектов} & \raisebox{1.5ex}{Размерность сцены} & Послед. реал. & Паралл. реал. \\
        \hline
        & \texttt{100x100} & 49 & 21 \\
        \cline{2-4}
        & \texttt{200x200} & 301 & 83  \\
        \cline{2-4}
        \texttt{2} & \texttt{500x500} & 1606 & 442 \\
        \cline{2-4}
        & \texttt{1000x1000} & 5068 & 1853 \\
        \cline{2-4}
        & \texttt{2000x2000} & 20428 & 8349 \\
        \hline
        & \texttt{100x100} & 135 & 49  \\
        \cline{2-4}
        & \texttt{200x200} & 380 & 168 \\
        \cline{2-4}
        \texttt{4} & \texttt{500x500} & 2024 & 672 \\
        \cline{2-4}
        & \texttt{1000x1000} & 7840 & 2672 \\
        \cline{2-4}
        & \texttt{2000x2000} & 32666 & 10859 \\
        \hline
        & \texttt{100x100} & 135 & 38 \\
        \cline{2-4}
        & \texttt{200x200} & 544 & 165 \\
        \cline{2-4}
        \texttt{8} & \texttt{500x500} & 3341 & 1072 \\
        \cline{2-4}
        & \texttt{1000x1000} & 13009 & 4233 \\
        \cline{2-4}
        & \texttt{2000x2000} & 51708 & 17864 \\
        \hline
        & \texttt{100x100} & 272 & 74 \\
        \cline{2-4}
        & \texttt{200x200} & 1098 & 302 \\
        \cline{2-4}
        \texttt{16} & \texttt{500x500} & 6693 & 1864 \\
        \cline{2-4}
        & \texttt{1000x1000} & 26108 & 7451 \\
        \cline{2-4}
        & \texttt{2000x2000} & 102328 & 31217 \\
        \hline
        \end{tabular}
	\end{center}
\end{table}

        Согласно данным, приведенным в таблице \ref{tbl:time}, время синтеза изображения зависит от размерности данного изображения как $O(m n)$ или $O(n^2)$ где m, n - размерности изображения.
        
        Также время синтеза изображения прямо пропорционально количеству объектов сцены, то есть зависит как $O(n)$, где $n$ - количество объектов сцены.

        Стоит отметить, что многопоточная реализация алгоритма оказалась  более эффективной при всех рассматриваемых размерностях изображения. Связано это с большими размерностями синтезируемых изображений (от 300 до 1000 пикселей). Для изображения малых размерностей, однопоточный алгоритм окажется более эффективным в связи с отсутствием дополнительных затрат по времени и памяти, требуемых для реализации многопоточности (создание потоков, совместный лоступ к ресурсам).

    \section{Вывод}

        В данном разделе было произведено экспериментально сравнение временных характеристик реализованного программного обеспечения.
        
        Время работы алгоритма имеет квадратичную зависимость от размерности синтезируемого изображения и линейную зависимость от количества объектов сцены.
        
        Наибоее эффективной по времени оказалась многопоточная реализация алгоритма. В среднем время ее работы меньше 3 раза, чем последовательной реализации.