\chapter*{ЗАКЛЮЧЕНИЕ}
\addcontentsline{toc}{chapter}{ЗАКЛЮЧЕНИЕ}

    В ходе курсового проекта было разработано программное обеспечение, предоставляющее возможность визуализации дисперсии света на прозрачных предметах. Разработанное программное обеспечение предоставляет функционал для задания материала и цвета поверхности, а так же задания и изменения в процессе работы положения точки наблюдения и источников света по их характеристикам (положению, интенсивности) в интерактивном режиме. В процессе выполнения данной работы были выполнены следующие задачи:

    \begin{itemize}
        \item изучение явления дисперсии с физической точки зрения;
        \item анализ существующих алгоритмов построения реалистичных изображений;
        \item выбор алгоритма для решения поставленной задачи;
        \item проектирование архитектуры и графического интерфейса программы;
        \item реализация структур данных и алгоритмов;
	    \item описание структуры разрабатываемого ПО;
	    \item написание программы и тестирование;
        \item исследование производительности программы.
    \end{itemize}