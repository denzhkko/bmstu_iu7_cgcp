\chapter*{ВВЕДЕНИЕ}
\addcontentsline{toc}{chapter}{ВВЕДЕНИЕ}

    Дисперсия - это разбиения белого пучка света на его цветные составляющие при прохождении через прозрачные поверхности. Наиболее известным экспериментом, показывающим это явление, является пропускание белого пучка света через призму и наблюдение светового спектра на экране. Явление дисперсии также иллюстрируют радуга и блеск драгоценных камней.

    Явление дисперсии занимало ума людей столетиями. Инженеры, проектирующие оптические приборы, стремились минимизировать ее проявление в своих приборах. В то время как ювелиры находились в постоянном стремлении преумножить блеск своих драгоценных камней.

    Дисперсия повсюду встречается в нашей жизни, она подробно изучена с физической точки зрения. При построении изображений, претендующих на фотореалистичность, нельзя не учитывать это явление.

    Целью моего проекта является разработка программного обеспечения для визуализации трехмерных объектов и наблюдения дисперсии света на прозрачных поверхностях c возможностью выбора пользователем объектов сцены из предложенного списка, а также задания источников освещения по их характеристикам: положению, интенсивности.

    Для достижения поставленной цели необходимо решить следующие задачи:

    \begin{itemize}
        \item изучение явления дисперсии с физической точки зрения;
        \item анализ существующих алгоритмов построения реалистичных изображений;
        \item выбор алгоритма для решения поставленной задачи;
        \item проектирование архитектуры и графического интерфейса программы;
        \item реализация структур данных и алгоритмов;
	    \item описание структуры разрабатываемого ПО;
	    \item написание программы и тестирование;
        \item исследование производительности программы.
    \end{itemize}
